\chapter{Forensic Analysis Methodology}

\section{Investigative Framework and Tools}
This investigation adhered to established forensic standards (ACPO Guidelines, ISO/IEC 27037, NIST SP 800-86) to ensure evidence integrity and reproducibility. The primary forensic tools employed were:

\begin{itemize}
    \item \textbf{Autopsy (version 4.22.1 for Windows)}: Core forensic analysis of filesystem artifacts, deleted content recovery, file carving, and metadata examination
    \item \textbf{Wireshark (version 4.4.6 for Windows)}: Network packet analysis and TCP session reconstruction
    \item \textbf{AccessData Registry Viewer (version 2.0.0.7)}: Registry analysis and user activity identification
    \item \textbf{John The Ripper (version 1.9.0)}: Password recovery for encrypted documents
    \item \textbf{ExifTool (version 12.30)}: Metadata analysis of image files
\end{itemize}

The analysis was conducted on a dedicated forensic workstation (AMD Ryzen 9 6900HX, 32GB DDR5 RAM, Windows 11 Pro 24H2) with hardware write-blockers and comprehensive logging to maintain evidence integrity. MD5 and SHA-256 hashing was employed throughout to verify evidence integrity.

\section{Specialized Analysis Techniques}

\subsection{Data Recovery and Artifact Extraction}
Key techniques employed for data recovery focused specifically on areas critical to answering the investigation objectives:

\begin{itemize}
    \item \textbf{File Carving}: Recovered deleted flight plan image from unallocated space using PhotoRec
    \item \textbf{Steganography Detection}: Identified and extracted hidden recipe data from image files
    \item \textbf{Password Recovery}: Successfully decrypted protected PDF and Excel files containing recipe information
    \item \textbf{Registry Analysis}: Extracted search history, Excel usage count, and URL information from NTUSER.DAT
\end{itemize}

\subsection{Network Activity Analysis}
Network traffic examination was conducted with specific focus on:

\begin{itemize}
    \item TCP session reconstruction between Taurus Smith's computer (192.168.1.158) and the unknown device
    \item Payload analysis of recipe transmission packets
    \item VMware virtualization detection through MAC address analysis (VMware\_b0:8d:62)
    \item Recovery of plaintext communications regarding Hawaii travel and steganography usage
\end{itemize}

\section{Evidence Correlation Methodology}

The strength of digital forensic analysis lies in correlation across multiple evidence sources. To establish a comprehensive case narrative, the following correlation methodologies were employed:

\begin{itemize}
    \item \textbf{Temporal Analysis}: Constructing precise timelines linking file creation, modification, network activity and registry entries
    \item \textbf{Attribution Analysis}: Connecting Ken Warren and Mike to specific files through metadata
    \item \textbf{Artifact Clustering}: Grouping related evidence (steganographic images, encrypted files) to identify operational patterns
    \item \textbf{Technical Signature Matching}: Identifying consistent technical approaches across different evidence sources
\end{itemize}

All findings were subjected to multiple validation checks to ensure accuracy, with hash verification performed throughout the analysis to maintain evidence integrity.