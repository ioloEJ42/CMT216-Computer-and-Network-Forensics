\chapter{Forensic Analysis Methodology}

\section{Investigative Framework and Standards}
This investigation follows an integrated framework combining established forensic methodologies from multiple authoritative sources:

\begin{itemize}
    \item \textbf{ACPO Guidelines for Digital Evidence}: Providing core principles for evidence handling and examination integrity
    \item \textbf{ISO/IEC 27037}: International standards for identification, collection, acquisition, and preservation of digital evidence
    \item \textbf{NIST SP 800-86}: Guidelines for integrating forensic techniques into incident response
\end{itemize}

This framework was selected to address the unique challenges presented by multi-source digital evidence analysis in corporate espionage cases, particularly where suspects may have employed anti-forensic measures to conceal activities.

\section{Forensic Analysis Technology and Configuration}

\subsection{Analysis Platform Configuration}
The primary forensic analysis was conducted using Autopsy (version 4.22.1 for Windows) and Wireshark (version 4.4.6 for Windows) on a dedicated forensic workstation with the following specifications:

\begin{itemize}
    \item CPU: AMD Ryzen 9 6900HX with Radeon Graphics (8 cores, 16 threads)
    \item RAM: 32GB DDR5 (2x16GB)
    \item Storage: 2TB NVMe SSD (primary analysis) + 20TB HDD Array (evidence storage)
    \item Network: Isolated forensic network with monitored connection points
    \item Operating System: Windows 11 Pro (24H2) with security hardening
\end{itemize}

The workstation was configured with application whitelisting, USB device control, and comprehensive activity logging to maintain the integrity of the analysis environment.

\subsection{Modular Forensic Examination Framework}
Autopsy's modular architecture was leveraged to implement a systematic examination strategy across multiple evidence categories:

\begin{table}[h]
\centering
\begin{tabular}{|p{5cm}|p{10cm}|}
\hline
\textbf{Module Category} & \textbf{Implemented Modules} \\
\hline
\textbf{File System Analysis} & 
\begin{itemize}
    \item File Type Identification
    \item Extension Mismatch Detector
    \item Interesting Files Identifier
    \item Data Source Integrity
\end{itemize} \\
\hline
\textbf{Content Analysis} & 
\begin{itemize}
    \item Keyword Search
    \item Picture Analyzer
    \item Email Parser
    \item Embedded File Extractor
    \item GPX Parser
    \item YARA Analyzer
\end{itemize} \\
\hline
\textbf{Activity Analysis} & 
\begin{itemize}
    \item Recent Activity
    \item Central Repository
    \item Targeted Timeline Analysis
\end{itemize} \\
\hline
\textbf{Specialized Analysis} & 
\begin{itemize}
    \item Encryption Detection
    \item PhotoRec Carver
    \item Virtual Machine Extractor
    \item Android Analyzer (aLEAPP)
    \item iOS Analyzer (iLEAPP)
\end{itemize} \\
\hline
\textbf{Verification} & 
\begin{itemize}
    \item Hash Lookup
    \item Data Source Integrity
\end{itemize} \\
\hline
\end{tabular}
\caption{Autopsy Modules Deployed for Forensic Examination}
\label{tab:autopsy_modules}
\end{table}

\subsection{Advanced Module Configuration}

\subsubsection{Encryption Detection Parameters}
The Encryption Detection module was configured with specific parameters optimized for identifying potentially obfuscated proprietary data:

\begin{itemize}
    \item Minimum entropy threshold: 7.5 (calibrated to detect encrypted content while minimizing false positives)
    \item Minimum file size: 5MB (focused on substantive encrypted containers)
    \item Enhanced detection options:
    \begin{itemize}
        \item Analysis of files with sizes that are multiples of 512 bytes (common encryption block size)
        \item Inclusion of slack space in analytical scope for partially overwritten encrypted data
    \end{itemize}
\end{itemize}

The entropy threshold selection balanced sensitivity against false positive rates, with validation testing performed using known-encrypted sample files to verify detection efficacy.

\subsubsection{Registry Analysis Configuration}
Specialized registry analysis of the NTUSER.DAT file utilized Registry Explorer (version 1.6.0) and RegRipper (version 3.0) with the following focus areas:

\begin{itemize}
    \item User search history extraction and pattern analysis
    \item Application execution frequency metrics with specific focus on Excel.exe usage
    \item Most recently typed URLs identification
    \item UserAssist record analysis for application usage patterns
    \item MRU (Most Recently Used) list examination for document access history
\end{itemize}

Custom RegRipper plugins were developed to extract specific recipe-related search terms and document access patterns relevant to the investigation's focus.

\section{Resource-Optimized Analysis Strategy}
Given the computational constraints and investigative priorities, a resource-optimized analytical approach was implemented to maximize evidential yield within available resources.

\subsection{Targeted Analysis Prioritization}
Rather than processing the entire evidence corpus with computationally intensive techniques, analysis was prioritized based on:

\begin{itemize}
    \item \textbf{User Directory Focus}: Analysis concentrated on user profiles and document repositories most likely to contain evidence relevant to the key investigative questions
    
    \item \textbf{Temporal Proximity}: Files and artifacts created or modified during time periods correlating with network activity between Smith's computer and the unauthorized device received priority analysis
    
    \item \textbf{File Type Prioritization}: Known steganography, encryption, and data exfiltration file formats received accelerated processing
    
    \item \textbf{Keyword Hit Concentration}: Areas with high concentrations of relevant keyword hits underwent more intensive scrutiny
\end{itemize}

\subsection{Iterative Analytical Process}
Analysis proceeded through defined phases with continual refinement:

\begin{enumerate}
    \item \textbf{Initial Triage}: Rapid assessment to identify high-value targets for detailed analysis
    
    \item \textbf{Focused Examination}: Detailed analysis of high-probability evidence sources identified during triage
    
    \item \textbf{Correlation Analysis}: Integration of findings across evidence sources to develop investigative hypotheses
    
    \item \textbf{Hypothesis Testing}: Targeted examination to validate or refine investigative theories
    
    \item \textbf{Comprehensive Documentation}: Thorough documentation of findings and their interpretive context
\end{enumerate}

This approach facilitated efficient resource allocation while ensuring thorough examination of potentially relevant evidence.

\section{Specialized Analytical Techniques}

\subsection{Advanced Data Recovery Methodology}
Specialized techniques were employed to recover potentially deleted or concealed data:

\begin{itemize}
    \item \textbf{File Carving}: Signature-based recovery across unallocated space using PhotoRec Carver module
    
    \item \textbf{Slack Space Analysis}: Examination of file slack for fragment recovery
    
    \item \textbf{File System Journal Analysis}: Recovery of deleted file metadata from NTFS journal records
    
    \item \textbf{Shadow Copy Examination}: Analysis of Volume Shadow Copies for previous file versions
    
    \item \textbf{Memory Structure Recovery}: Reconstruction of file system structures to recover deleted directory entries
\end{itemize}

\subsection{Anti-Forensic Countermeasure Analysis}
The investigation incorporated specialized techniques to counter potential anti-forensic methods:

\begin{itemize}
    \item \textbf{Steganography Detection}: Statistical and visual analysis of media files for hidden content
    
    \item \textbf{Signature Verification}: Identification of files with mismatched headers and extensions
    
    \item \textbf{Timestamp Analysis}: Detection of manipulated file system timestamps through consistency checking
    
    \item \textbf{Embedded Content Extraction}: Recovery of hidden data within container files
    
    \item \textbf{Metadata Inconsistency Detection}: Identification of manipulated or sanitized metadata
\end{itemize}

\subsection{Network Forensics Methodology}
Analysis of network communications employed a structured protocol-based analytical framework:

\begin{itemize}
    \item \textbf{Packet Capture Analysis}: Using Wireshark (version 3.6.2) for TCP/IP session reconstruction
    
    \item \textbf{Session Correlation}: Linking network timestamps with host-based artifacts
    
    \item \textbf{Protocol Analysis}: Examination of protocol-specific behaviors and anomalies
    
    \item \textbf{Traffic Pattern Identification}: Detection of data exfiltration signatures
    
    \item \textbf{Encryption Analysis}: Identification and classification of encrypted communications
\end{itemize}

Custom Wireshark display filters were created to isolate communication patterns specific to recipe transfer and data exfiltration scenarios.

\section{Cross-Evidence Correlation and Synthesis}
To develop a comprehensive understanding of the case, multiple correlation methodologies were employed to integrate findings across evidence sources:

\begin{itemize}
    \item \textbf{Temporal Correlation}: Linking file system activities with network events to establish activity timelines
    
    \item \textbf{Entity Relationship Mapping}: Connecting user accounts, files, and network communications
    
    \item \textbf{Behavioral Analysis}: Identifying patterns across different evidence sources that indicate specific user activities
    
    \item \textbf{Artifact Cross-Referencing}: Validating findings through identification of the same information across multiple data sources
    
    \item \textbf{Contextual Analysis}: Interpreting technical findings within the case-specific context
\end{itemize}

This multi-faceted analytical approach ensured that individual findings were considered within their broader investigative context, enabling more accurate interpretation of the evidence as a whole.

The comprehensive methodology detailed in this chapter provided the framework for the systematic forensic analysis documented in subsequent sections. Each aspect of the methodology was selected to maximize the recovery and interpretation of evidence while maintaining strict adherence to forensic best practices and legal requirements.