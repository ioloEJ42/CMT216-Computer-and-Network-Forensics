\chapter{NTUSER.DAT Registry Analysis}

\section{Registry Evidence Extraction}
The NTUSER.DAT registry hive was analyzed using FTK Imager (Version 4.7.1) and RegRipper (Version 3.0) to extract user activity data, search history, and application usage patterns. This analysis directly addresses investigation question 5 regarding user searches, Excel usage, and the latest typed URL.

\section{Search Activity Analysis}
The WordWheelQuery registry key (LastWrite: 2012-04-04 15:45:18Z) revealed searches directly related to donut research and potential corporate espionage:

\begin{table}[htbp]
    \centering
    \begin{tabular}{|l|p{10cm}|l|}
        \hline
        \textbf{Query} & \textbf{Forensic Significance} & \textbf{Order} \\
        \hline
        "donut" & Basic product research & 1 \\
        \hline
        "newrecipe" & Possible product development & 2 \\
        \hline
        "Ms.Taurus" & Username reference (matches suspect's profile) & 3 \\
        \hline
        "accounts" & Possible financial or user account access & 4 \\
        \hline
        "cardiff donut" & Location-specific business research & 5 \\
        \hline
        "best donut" & Competitive analysis & 6 \\
        \hline
    \end{tabular}
    \caption{Search queries from NTUSER.DAT showing donut research progression}
    \label{tab:search_queries}
\end{table}

\textbf{Registry Path:} \texttt{NTUSER.DAT\textbackslash Software\textbackslash Microsoft\textbackslash Windows\textbackslash CurrentVersion\textbackslash Explorer\textbackslash WordWheelQuery}

The progression from general donut research to location-specific queries focused on Cardiff directly corroborates evidence found in the network communications and flight plan.

\section{Excel Usage Analysis}
The UserAssist registry key provided definitive evidence regarding Excel usage:

\begin{itemize}
    \item \textbf{Excel Execution Count:} \underline{5 distinct instances}
    \item \textbf{Last Execution:} 2012-04-04 15:43:14Z
    \item \textbf{Excel Version:} Office 2010 (Office14)
\end{itemize}

\textbf{Registry Path:} \texttt{NTUSER.DAT\textbackslash Software\textbackslash Microsoft\textbackslash Windows\textbackslash CurrentVersion\textbackslash Explorer\textbackslash UserAssist}\\
\textbf{Entry:} \texttt{\{7C5A40EF-A0FB-4BFC-874A-C0F2E0B9FA8E\}\textbackslash Microsoft Office\textbackslash Office14\textbackslash EXCEL.EXE (5)}

This finding directly answers question 5 regarding Excel usage frequency, confirming exactly 5 executions of Excel.exe.

\section{Latest Typed URL Identification}
Based on the TypedURLs registry key (LastWrite: 2012-04-03 22:37:55Z) and MRU order, the most recently typed URL was:

\begin{center}
\textbf{\url{http://cardiffdonut.com/fwlink/?LinkId=69157}}
\end{center}

\textbf{Registry Path:} \texttt{NTUSER.DAT\textbackslash Software\textbackslash Microsoft\textbackslash Internet Explorer\textbackslash TypedURLs}

This URL contains a domain name directly related to donuts in Cardiff and uses Microsoft's forwarding link pattern (?LinkId=). This finding directly answers the third part of question 5 regarding the latest typed URL.

\section{Additional Suspicious Registry Artifacts}

\subsection{Timeline of Key Activities}
Registry LastWrite timestamps provide a chronological sequence of suspect activities:

\begin{table}[htbp]
    \centering
    \begin{tabular}{|p{4cm}|p{9cm}|}
        \hline
        \textbf{Timestamp (UTC)} & \textbf{Activity} \\
        \hline
        2012-04-03 22:12:41Z & Execution of suspicious "C:\textbackslash dllhot.exe" \\
        \hline
        2012-04-03 22:37:55Z & Direct access to IP 131.10.28.251:53 and cardiffdonut.com \\
        \hline
        2012-04-04 15:43:14Z & Excel last execution \\
        \hline
        2012-04-04 15:45:18Z & Final search queries performed \\
        \hline
        2012-04-04 15:52:45Z & Command prompt (cmd.exe) last executed \\
        \hline
    \end{tabular}
    \caption{Activity timeline from registry LastWrite times}
    \label{tab:activity_timeline}
\end{table}

\subsection{Notable Security Indicators}
Several registry artifacts indicated sophisticated operational security measures:

\begin{itemize}
    \item Execution of non-standard application "dllhot.exe" from root directory
    \item Direct access to an IP address with DNS port (53), suggesting potential tunneling
    \item Access to folders with suspicious names: "Agents-List-CLASSIFIED-TOP-SECRET" and "CC R\&D Backstopped Accounts"
    \item Browser switching between Internet Explorer and Firefox, suggesting compartmentalized activities
    \item Command-line operations as the final activity, potentially for cleanup
\end{itemize}

\textbf{Registry Path:} \texttt{NTUSER.DAT\textbackslash Software\textbackslash Microsoft\textbackslash Windows\textbackslash CurrentVersion\textbackslash Explorer\textbackslash RecentDocs\textbackslash Folder} (LastWrite: 2012-04-04 15:43:17Z)

\section{Registry Analysis Conclusions}
This analysis of the NTUSER.DAT registry hive directly answers question 5 from the investigation requirements:

\begin{itemize}
    \item \textbf{User Search History}: Searches focused on donuts, recipes, and specifically Cardiff-based businesses
    \item \textbf{Excel.exe Usage Count}: Excel was executed \underline{exactly 5 times}
    \item \textbf{Latest Typed URL}: \url{http://cardiffdonut.com/fwlink/?LinkId=69157}
\end{itemize}

Additionally, the registry evidence corroborates findings from other investigation sections by:

\begin{itemize}
    \item Confirming Cardiff connection through search history and URL access
    \item Demonstrating intentional research into donut recipes
    \item Revealing suspicious software usage and command-line operations
    \item Establishing a timeline consistent with data exfiltration activities
\end{itemize}