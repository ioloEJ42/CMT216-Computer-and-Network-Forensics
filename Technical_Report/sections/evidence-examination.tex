\chapter{Evidence Examination}

\section{File System Structure Analysis}
The file system structure analysis is a pivotal step in the forensic examination process. It involves a comprehensive review of the file system hierarchy, allocation tables, and directory structures within the digital evidence. This analysis aims to reconstruct the user's activities and identify the locations where data might be intentionally hidden or inadvertently left behind.

Figure 1, labelled as 'File System Analysis Of File System Type', showcases a table detailing the partition structure of the digital evidence. It indicates two partitions: an unallocated space labelled as 'vol1 (Unallocated: 0-63)' and a Win95 FAT32 (LBA) file system labelled as 'vol2 (Win95 FAT32: 0x0c: 63-3915584)'. The presence of unallocated space could be significant, as it may contain remnants of previously deleted files or data fragments not currently linked to any file in the directory structure. The FAT32 file system partition is substantial in size and is the primary focus for further examination due to its allocation.

Figure 2.1, titled 'File Types', provides a pie chart illustrating the distribution of file types within the analysed digital evidence. Notably, the chart highlights a significant portion of space utilized by 'Other' files, which may include less common file types or possibly encrypted or obfuscated files requiring further scrutiny. The 'Documents' and 'Executables' categories are also of considerable size, potentially containing user-created content and installed software, respectively. These categories will be vital in the search for evidence related to the key investigative questions.

The file system structure analysis, supported by Figure 1 and Figure 2.1, has yielded vital insights into the composition and potential areas of interest within the digital evidence. The identified unallocated space necessitates a meticulous examination to recover any artefacts that may have been intentionally deleted or lost over time. The FAT32 file system, widely known for its usage in removable storage devices, must be explored thoroughly, with particular attention paid to document and executable files that may be pertinent to the case. The category labelled 'Other' within the pie chart suggests the presence of files that do not fit typical file type categories, which warrants a deeper investigation as these could be employed to conceal sensitive data. Moving forward, each file type represented will be analysed by forensic best practices to ensure no evidence is overlooked.

\begin{table}[htbp]
\centering
\begin{tabular}{|p{3cm}|p{10cm}|p{4cm}|}
\hline
\textbf{Directory Name} & \textbf{Notes} & \textbf{Full Path} \\
\hline
All Users & Contains shared application data; checks for installed software related to the case. & /Documents and Settings/All Users \\
\hline
Application Data & May contain user-specific application usage data; look for evidence of data transfer tools. & /Documents and Settings/[User]/Application Data \\
\hline
Desktop & Often contains downloaded files or shortcuts to important documents. & /Documents and Settings/[User]/Desktop \\
\hline
My Documents & A common location for storing personal files; investigate for any work-related documents. & /Documents and Settings/[User]/My Documents \\
\hline
My Music & Not typically relevant, but check for audio recordings related to the case. & /Documents and Settings/[User]/My Music \\
\hline
My Pictures & Look for images that may contain steganography or screenshots of sensitive information. & /Documents and Settings/[User]/My Pictures \\
\hline
donutPics & Highly relevant due to case context; inspect for images of the secret recipe. & /Documents and Settings/Taurus Smith/donutPics \\
\hline
hideme & The name suggests concealment; prioritize analysis for hidden or encrypted files. & /Documents and Settings/Taurus Smith/hideme \\
\hline
tools & Could contain software used for data hiding or encryption; & /Documents and Settings/Taurus Smith/tools \\
\hline
Family Photos & Could be innocuous but validate for mislabeled sensitive data. & /Documents and Settings/Taurus Smith/Family Photos \\
\hline
important email.eml & Direct relevance suspected; analyze contents for communications regarding espionage. & /Documents and Settings/Taurus Smith/important email.eml \\
\hline
Recycler & Check for recently deleted files that may have been disposed of to hide activities. & /Recycler \\
\hline
Program Files & Standard directory; investigate for unauthorized or suspicious installations. & /Program Files \\
\hline
WINDOWS & System directory; examine for any unusual modifications or access. & /WINDOWS \\
\hline
\end{tabular}
\caption{Directory Tree}
\label{table:directory-tree}
\end{table}

\textbf{Directory Tree Notes:}\\
Upon examining the directory tree outlined in Table \ref{table:directory-tree}, several directories of interest have been identified that may hold evidence pertinent to the case. The donutPics directory under Taurus Smith's user profile is of particular importance due to its potential to contain images of the secret recipe, given the context of the investigation. The hide and tools directories also suggest a deliberate attempt to conceal activities or information, which warrants a thorough analysis of any hidden or encrypted files. Additionally, multiple instances of important email.eml files across various user directories suggest communication activities that may be related to the alleged corporate espionage and must be carefully examined. The presence of user-specific folders such as My Documents, My Pictures, and the Recycler bin are standard yet could contain inadvertently stored sensitive information or clues to intentional data deletion. The structured approach in Table \ref{table:directory-tree} provides a roadmap for prioritizing and methodically analyzing these directories to uncover the full scope of the suspect's activities.

\section{Examination of Unallocated Space}
Unallocated space on a storage device refers to the areas not currently associated with a file or a file system. Data in these areas often includes remnants of deleted files, making it a rich source for potential evidence in forensic investigations.

\textbf{Unallocated Space Findings:}\\
Two significant files were discovered in the unallocated space of the image from Taurus Smith's laptop:
\begin{enumerate}
    \item File 1:
    \begin{itemize}
        \item Name: Unalloc\_8524\_1003921920\_1979842560
        \item Location: /img\_Taurus Laptop.001/vol\_vol2/\$Unalloc/Unalloc\_8524\_1003921920\_1979842560
        \item Size: 975764480 bytes
        \item Type: Unallocated Blocks
        \item MIME Type: application/octet-stream
        \item The content and relevance of this file to the investigation remain to be analyzed.
    \end{itemize}
    
    \item File 2:
    \begin{itemize}
        \item Name: Unalloc\_8524\_43768320\_1003921408
        \item Location: /img\_Taurus Laptop.001/vol\_vol2/\$Unalloc/Unalloc\_8524\_43768320\_1003921408
        \item Size: 821198336 bytes
        \item Type: Unallocated Blocks
        \item MIME Type: application/octet-stream
        \item This file contains text hidden in the hex unveiling another hidden recipe.
    \end{itemize}
\end{enumerate}

\begin{table}[htbp]
\centering
\begin{tabular}{|p{4cm}|p{8cm}|}
\hline
\textbf{Metadata Field} & \textbf{Value} \\
\hline
Name & /img\_Taurus Laptop.001/vol\_vol2/\$Unalloc/Unalloc\_8524\_43768320\_1003921408 \\
\hline
Type & Unallocated Blocks \\
\hline
MIME Type & application/octet-stream \\
\hline
Size & 821198336 \\
\hline
File Name Allocation & Unallocated \\
\hline
Metadata Allocation & Unallocated \\
\hline
Modified & 0000-00-00 00:00:00 \\
\hline
Accessed & 0000-00-00 00:00:00 \\
\hline
Created & 0000-00-00 00:00:00 \\
\hline
Changed & 0000-00-00 00:00:00 \\
\hline
MD5 & Not calculated \\
\hline
SHA-256 & Not calculated \\
\hline
Hash Lookup Results & UNKNOWN \\
\hline
Internal ID & Internal ID 8525 \\
\hline
\end{tabular}
\caption{Metadata for File 2}
\label{table:metadata-file2}
\end{table}

\textbf{Analysis of Metadata:}\\
\begin{itemize}
    \item \textbf{Time Stamps:} All timestamps being set to '0000-00-00 00:00:00' indicates either a wiping of the metadata or a system error. This is common in unallocated space where file system metadata is not always preserved.
    \item \textbf{Size:} The significant size of the file suggests it may have been a large document or a collection of data.
    \item \textbf{MIME Type:} The MIME type being 'application/octet-stream' indicates a generic binary file, common for files in unallocated space where the original file type isn't recognized.
\end{itemize}

\textbf{Location of Hidden Text:}\\
The hidden recipe text in the second file was possibly within the binary data, requiring data carving techniques to extract. The lack of file system allocation suggests that the original file containing the recipe was deleted, with remnants persisting in the unallocated space.

\section{Operating System Analysis}
\begin{table}[htbp]
\centering
\begin{tabular}{|p{6cm}|p{8cm}|}
\hline
\textbf{Attribute} & \textbf{Value} \\
\hline
Temporary Files Directory & \%SystemRoot\%\textbackslash{}TEMP \\
\hline
Source File Path & /img\_Taurus Laptop.001 \\
\hline
Program Name & Microsoft Windows XP Service Pack 2 \\
\hline
Product ID & 55274-337-8535232-22871 \\
\hline
Processor Architecture & x86 \\
\hline
Path & D:\textbackslash{}WINDOWS \\
\hline
Owner & ADXP \\
\hline
Name & FRODO1 \\
\hline
Artifact ID & -9223372036854775334 \\
\hline
\end{tabular}
\caption{Operating System Information}
\label{table:os-info}
\end{table}

\textbf{Analysis of Operating System Information:}\\
The operating system information from Taurus Smith's laptop, running Microsoft Windows XP Service Pack 2, offers crucial insights into the user environment and system configuration. The presence of Windows XP, an older operating system, might indicate either a lack of system updates or a preference for a familiar environment by the user. The temporary files directory (\%SystemRoot\%\textbackslash{}TEMP) is a standard location, often scrutinized for remnants of recent activity, temporary files, or artefacts left by programs.

The processor architecture being x86 suggests compatibility with a wide range of software, possibly including older or less sophisticated applications. The path to the Windows directory (D:\textbackslash{}WINDOWS) is standard, but noting the drive letter can be important in understanding the system's storage structure.

The owner of the system is identified as 'ADXP', and the computer name is 'FRODO1'. These details could be pivotal in correlating system activities with a specific user or in understanding the network environment if the device was part of a larger infrastructure.

Lastly, the product ID provides a unique identifier for the operating system, which could be useful in licensing investigations or verifying the system's authenticity.

\textbf{Implications:} This operating system information not only helps in constructing the digital environment of the suspect but also aids in pinpointing specific user activities and system configurations. Understanding the operating system's setup is crucial for identifying how data was managed, stored, or potentially concealed. It also provides a foundation for further investigation into user accounts, installed applications, and system logs, all of which can yield valuable information in a forensic investigation.
