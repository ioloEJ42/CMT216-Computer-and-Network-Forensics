\chapter{Evidence Acquisition and Preservation}

\section{Initial Evidence Assessment}
The digital evidence central to this investigation encompasses multiple device types with varying states of viability. Following ACPO Principle 1 that "No action taken by law enforcement agencies, persons employed within those agencies or their agents should change data which may subsequently be relied upon in court," a systematic initial assessment was conducted to determine appropriate handling procedures for each evidence item.

\subsection{Device Inventory and Condition Analysis}
The evidence items submitted for forensic examination included:

\begin{itemize}
    \item \textbf{USB Flash Drive}: Containing an image of Taurus Smith's laptop hard drive, received in good physical condition with no visible damage. This device represented the primary evidentiary source for the investigation.
    
    \item \textbf{Mobile Device (Exhibit A)}: A smartphone with extensive liquid damage affecting both the exterior casing and, presumably, internal components. The severity of damage indicated that traditional acquisition methods would likely prove ineffective. While specialized recovery services might potentially recover data from this device, this investigation focused primarily on the USB flash drive contents given resource constraints and evidential priorities.
    
    \item \textbf{Network Capture Files (Exhibits B-E)}: Digital packet captures documenting communications between Taurus Smith's workstation and an unidentified device on the company network. These files were received in their original digital format with no integrity concerns noted.
    
    \item \textbf{NTUSER.DAT File}: A Windows registry hive file recovered from a company computer, potentially linked to Taurus Smith's activities. The file was received intact with no corruption indicators.
\end{itemize}

Following ACPO Principle 2, all evidence handling was performed only by qualified forensic personnel with appropriate documentation of actions taken. A comprehensive inventory was created documenting the physical characteristics, connection interfaces, storage capacities, and apparent condition of each item upon receipt.

\section{Specialized Acquisition Procedures}
The acquisition of digital evidence adhered to forensically sound methodologies designed to maintain evidential integrity while maximizing the recovery of potentially relevant data.

\subsection{USB Flash Drive Verification}
Prior to analytical processing, the USB flash drive image was authenticated using a multi-hash verification protocol employing three distinct cryptographic algorithms:

\begin{table}[h]
\centering
\begin{tabular}{|p{3cm}|p{8cm}|p{3cm}|}
\hline
\textbf{Hash Algorithm} & \textbf{Hash Value} & \textbf{Verification Status} \\
\hline
MD5 & \texttt{56aeba1a708c5210c8728e5a2560f9ca}  & Verified \\
\hline
SHA-1 & \texttt{3b023acd0e09d7db8bf5d1df725135a5}\\\texttt{f3bfc481} & Verified \\
\hline
SHA-256 & \texttt{a2f49fa7ce6b111c6e198de2ca4a24a8}\\\texttt{e73d6d85291805db5bede4d60fab23be}  & Verified \\
\hline
\end{tabular}
\caption{Hash Verification Results for USB Drive Image}
\label{tab:hash_verification_acq}
\end{table}

This multi-hash approach provides enhanced authentication assurance through the mathematical improbability of hash collisions across different algorithms. Each hash value was calculated at regular intervals throughout the investigation to verify continued data integrity.

\subsection{Mobile Device Recovery Assessment}
The liquid-damaged mobile device (Exhibit A) presented significant acquisition challenges. A non-invasive preliminary assessment indicated:

\begin{itemize}
    \item Corrosion on external connection ports
    \item Visible liquid ingress indicators on internal components
    \item Non-responsive power-on state when connected to external power sources
\end{itemize}

These factors indicated that chip-off forensics or specialized recovery services would be required for data extraction. Given the investigation's resource constraints and the availability of alternative evidence sources, full recovery from this device was deferred, though it remains available for future examination if deemed necessary.

\section{Evidence Handling and Preservation Infrastructure}
In accordance with ACPO Principle 3 requiring a comprehensive audit trail of all processes, robust evidence management protocols were established to maintain evidential integrity.

\subsection{Physical Security Measures}
All digital evidence was secured in a controlled environment with:

\begin{itemize}
    \item Restricted access using biometric authentication
    \item Continuous video monitoring of the forensic workspace
    \item Anti-static workstations with appropriate grounding
    \item Environmental controls (temperature: 20°C ± 2°, humidity: 40\% ± 5\%)
    \item Faraday-shielded examination areas to prevent remote network access or wiping attempts
\end{itemize}

\subsection{Digital Evidence Storage Protocol}
Working copies of evidential materials were stored on a dedicated forensic storage system featuring:

\begin{itemize}
    \item RAID-5 configuration ensuring data redundancy
    \item Write-once media for critical evidence preservation
    \item Encrypted storage volumes with multi-factor authentication
    \item Automated hash verification monitoring
    \item Uninterruptible power supply protection
\end{itemize}

\subsection{Documentation and Chain of Custody}
Following ACPO Principle 4 regarding overall responsibility for adherence to legal principles, comprehensive documentation was maintained throughout the acquisition process:

\begin{itemize}
    \item Contemporaneous notes with timestamped entries
    \item Photographic documentation of physical evidence
    \item Detailed logging of all access events and procedures
    \item Command-line history and tool configuration settings
    \item Error logs and anomaly documentation
\end{itemize}

The chain of custody system provided comprehensive traceability for all evidentiary items, ensuring that each access instance was properly documented and justified. This documentation framework ensures that the investigation's findings can be independently verified and reproduced by third parties, as required by ACPO Principle 3.

This rigorous evidence acquisition and preservation methodology laid the foundation for the detailed forensic analysis described in subsequent chapters. By establishing secure baselines for evidential integrity, the investigation maintained compliance with established forensic principles while ensuring optimal recovery of potentially relevant data.